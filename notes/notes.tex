\documentclass{article}
\usepackage{amsmath, amssymb, dsfont, physics}
\usepackage[margin=1.5cm]{geometry}


\title{Notes}
\author{SCK team}
\date{}

\begin{document}
    \maketitle

\section{Notation}
\begin{itemize}
    \item Average emission per source: $x_1,\dots,x_m$ where $m = 200$.
    \item Measured concentrations $y_1,\dots,y_n$ where $n = 4636$. These are consecutive measurements for several measurement stations. 
    \item $M_{i,(j,\Delta)}$ is a block in the sensitivity matrix $M$ that represents the contribution of $x_j$ from $\Delta - 1$ days ago to the observed concentration $y_i$.
    \item Scaling factors $s_1,\dots,s_m$ (in \texttt{scalings.csv}). These are put in a diagonal matrix $S$.
    \item $\vb{1}_q = [1\,\, \dots\,\, 1]^T \in \mathbb{R}^q$ is the constant vector consisting of only ones.
    \item $\otimes$ is the Kronecker product. In particular,
        $$
        \underbrace{
        \begin{bmatrix}
            a_1 \\
            \vdots \\
            a_p
        \end{bmatrix}
        }_{\in \mathbb{R}^{p}}
        \otimes 
        \underbrace{
        \begin{bmatrix}
            1 \\
            \vdots \\
            1
        \end{bmatrix}
        }_{\in \mathbb{R}^{q}}
        =
        \underbrace{
        \begin{bmatrix}
            a_1 \\
            \vdots \\
            a_1 \\
            a_2 \\
            \vdots \\
            a_2 \\
            \vdots \\
            a_p
        \end{bmatrix}
        }_{\in \mathbb{R}^{pq}}
        .$$
\end{itemize}

\section{Model}
With $n=1,\dots,4636$ and $m = 1,\dots,200$, the initial model is
$$
y_i \approx
\hat{y}_i = \sum_{j=1}^m \sum_{\Delta = 1}^{15}  M_{i,(j,\Delta)} \frac{x_{j}}{s_j}
.$$
That is, for each emission source $j$, we simulate the contribution of 15 days, and add this together for all $j$. A matrix formulation of this reads

\begin{align*}
\begin{bmatrix}
    \hat{y}_1 \\ 
    \vdots \\
    \hat{y}_{m}
\end{bmatrix}
&=
\begin{bmatrix}
    M_{1,(1,1)} & M_{1,(1,2)} & \cdots & M_{1,(n,15)} \\
    \vdots & \vdots & & \vdots \\
    M_{m,(1,1)} & M_{m,(1,2)} & \cdots & M_{m,(n,15)} \\
\end{bmatrix}
\begin{bmatrix}
    s_1^{-1} x_1\\
    \vdots \\
    s_1^{-1} x_1 \\
    s_2^{-1} x_2 \\
    \vdots \\
    s_n^{-1} x_n
\end{bmatrix} \\
&=
\begin{bmatrix}
    M_{1,(1,1)} & M_{1,(1,2)} & \cdots & M_{1,(n,15)} \\
    \vdots & \vdots & & \vdots \\
    M_{m,(1,1)} & M_{m,(1,2)} & \cdots & M_{m,(n,15)} \\
\end{bmatrix}
\left( 
     \begin{bmatrix}
        s_1 & & \\
         & \ddots & \\
        & & s_n
     \end{bmatrix}^{-1}
     \begin{bmatrix}
        x_1 \\
        \vdots \\
        x_n
     \end{bmatrix}
     \otimes \vb{1}_{15}
\right)
\end{align*}
where $s_i$ are the scalings.

Now we present a different interpretation. Instead of $S^{-1} x \otimes \vb{1}_{15}$, we make a vector $\tilde{x} := S^{-1} x \otimes \vb{1}_{365}$. This is a vector of length $365 \times 200$ that gives the emission from day 1 to day 365 for each source. This is assumed to be constant for all days. Therefore, $\tilde{x}$ consists of $m = 200$ blocks, each of which is constant.

With this formulation, the predicted concentration in station $y_i$ is 
$$
        \hat{y}_i
    =
    \begin{bmatrix}
        \cdots & M_{1,(1,1)} & \dots & M_{1,(1,15)} & 0 & \dots & M_{1,(2,1)} & \dots & M_{1,(2,15)} & \dots
    \end{bmatrix}
    (S^{-1} x \otimes \vb{1}_{365})
.
$$

Now assume that in each station, there is one observation per day, named $y_{kt}$, where $k = 1,\dots,K$ where $K$ is the number of stations and $t = 1,\dots,365$. The index $t=1$ corresponds to the most recent measurement. Then 
$$
\begin{bmatrix}
    \hat{y}_{1,1} \\
    \hat{y}_{1,2} \\
    \vdots \\
    \hat{y}_{2,1} \\ 
    \vdots \\
    \hat{y}_{K,365}
\end{bmatrix}
=
\underbrace{
\begin{bmatrix}
    M_{(1,1), (1,1)} & M_{(1,1), (1,2)} & \cdots & M_{(1,1),(1,15)} & 0 & \cdots & \cdots & M_{(1,1), (2,1)} & \cdots & 0 \\
    0 & M_{(1,2), (1,1)} & \cdots & M_{(1,2), (1,14)} & M_{(1,2), (1,15)} & 0 & \cdots & 0 & \cdots & 0 \\
    \vdots & \vdots & & \vdots & \vdots & \vdots & & \vdots & & \vdots \\
    M_{(2,1), (1,1)} & M_{(2,1), (1,2)} & \cdots & M_{(2,1),(1,15)} & 0 & \cdots & \cdots & M_{(2,1), (2,1)} & \cdots & 0 \\
    \vdots \\
    \vdots
\end{bmatrix}
}_{ := M_{shift}}
\tilde{x}
$$
where 
$$
\tilde{x} := S^{-1} x \otimes \vb{1}_{365}
=
\begin{bmatrix}
    s_1^{-1} x_1 \\
    \vdots \\
    s_1^{-1} x_1 \\ 
    s_2^{-1} x_2 \\
    \vdots \\ 
    s_m^{-1} x_m
\end{bmatrix}
.$$
If we weigh each day and each emission source by some factor, we get
$$
\hat{y} = M_{shift} W (S^{-1} x \otimes \vb{1}_{365})
$$
where $W$ is a diagonal matrix containing the weights.
If $W$ is what we should be looking for, then we can use well-known numerical techniques for solving linear matrix equations. For theoretical purposes, we describe the problem as a large linear system involving Kronecker products.
Writing $W = \mathrm{diag}(w)$ for some vector $w$, the above can be reformulated as a liner operation applied to $w$ as follows:
$$
\hat{y} = ((S^{-1} x \otimes \vb{1}_{365})^T \otimes M_{shift}) \mathrm{vec}(\mathrm{diag}\, w)
$$
where $\mathrm{vec} \circ \mathrm{diag}$ is a constant linear map that can be inferred from the dimensions of the problem.

\end{document}

